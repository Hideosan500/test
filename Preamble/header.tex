% !TeX encoding = UTF-8
% !TeX spellcheck = en_US
% !TeX root = ../Thesis.tex

%* --------------------------------------------------------------------------------	*
%* "THE BEER-WARE LICENSE" (Revision 42):											*
%* <weigner.thomas@gmail.com> wrote this file. As long as you retain this notice	*
%* you can do whatever you want with this stuff. If we meet some day, and you think	*
%* this stuff is worth it, you can buy me a beer in return.   Thomas Weigner		*
%* --------------------------------------------------------------------------------	*

% *****************************************************************************
% ******************************** Settings ***********************************
% *****************************************************************************
%BEGIN_FOLD
% ****************************** Miscallaneous ********************************
%BEGIN_FOLD

%END_FOLD

% ************ Custom Fonts (like different typewriter fonts etc.) ************
%BEGIN_FOLD
% Add `customfont' in the document class option to use this section
\ifsetCustomFont
% Set your custom font here and use `customfont' in options. Leave empty to
% load computer modern font (default LaTeX font).
	\RequirePackage{helvet}

%TODO what does this?
%---For use with XeLaTeX
  \setmainfont[
    Path              = ./libertine/opentype/,
    Extension         = .otf,
    UprightFont = LinLibertine_R,
    BoldFont = LinLibertine_RZ, % Linux Libertine O Regular Semibold
    ItalicFont = LinLibertine_RI,
    BoldItalicFont = LinLibertine_RZI, % Linux Libertine O Regular Semibold Italic
  ]
  {libertine}
  % load font from system font
  \newfontfamily\libertinesystemfont{Linux Libertine O}
\fi
%END_FOLD

% ******************************** Formatting *********************************
%BEGIN_FOLD
%---Set margins
\ifsetCustomMargin		% add `custommargin' in the document class options to set options for geometry package here or load different package
	\geometry{paper=\papersize, hmarginratio=1:1, vmarginratio=1:2, scale=0.7, bindingoffset=\bindingOffset}	% uses geometry package
% 	\geometry{left=37mm,right=30mm,top=35mm,bottom=30mm}														% uses geometry package
\else
%TODO change format when notes are present
%	\ifenabledTodonotes	% set layout compatible with todonotes			
%		\geometry{paper=\papersize,left=15mm,right=45mm,top=30mm,bottom=50mm}	% more margin on the side for notes
%	\else			% set layout via Koma script class as default
		\KOMAoptions{headinclude=true}											% include header into text body
		\KOMAoptions{footinclude=false}											% include footnotes into text body
		\KOMAoptions{BCOR=\bindingOffset}										% pass binding correction
		\KOMAoptions{DIV=12}													% calculated or # of div per page to calculate typearea with package typearea
%	\fi
\fi

%---Build header after typearea is defined
\setHeaderandFooter						% customization can be done in file header_and_footer_style.tex	

%---Set how the last line of a page is aligned 
%\flushbottom							% at the bottom
\raggedbottom							% where ever the last line 'falls'

%---Add spaces between paragraphs
\setlength{\parskip}{0.5em}

%---To remove the excess top spacing for enumeration, list and description
\usepackage{enumitem}
\setlist[enumerate,itemize,description]{topsep=0em}

%---Don't break enumeration (etc.) across pages in an ugly manner (default 10000)
\clubpenalty=500
\widowpenalty=500
\usepackage[perpage]{footmisc}			% Range of footnote options

%---define lengths
\newlength{\Afour}
\setlength{\Afour}{210cm}
%END_FOLD

% ******************************* Bibliography ********************************
%BEGIN_FOLD
% add `custombib' in the document class option to use this section
% load one bibligaraphy LaTeX package (natbib, BibTeX,...)
\ifuseCustomBib
%	\RequirePackage[square, sort, numbers, authoryear]{natbib}
%	\setNatBibtrue \makeatletter\let\chapter\@undefined\makeatother		% set natbib true and natbib fix for todonotes list
	\usepackage[backend=\BibBackend, style=numeric, citestyle=numeric, sorting=none, natbib=true]{biblatex} \setBibLaTeXtrue
\fi

%---if loaded tell biblatex where the .bib file is (has to be in the preamble)
\ifsetBibLaTeX
	\bibliography{\bibliographyPath}
\else
	\newcommand{\autocite}{\cite}		% replace autocite with cite if BibLaTeX is not used
\fi

%---further BibLaTeX setup
\ifsetBibLaTeX
	%---set additional keys
	\ExecuteBibliographyOptions{
		bibencoding=utf8,
		maxcitenames=1,
		mincitenames=1,
		maxbibnames=5,
		minbibnames=3,
		uniquename=full,
		ibidtracker=strict,
		uniquelist=minyear,
		labeldateparts=true,
		clearlang=true,
%		indexing=bib,
%		block=nbpar,
		hyperref=true,
		isbn=false,
		eprint=false,
		url=false,
		arxiv=pdf
	}
	%---define subbibliographies headings
	\ifBibSections
		\ExecuteBibliographyOptions{refsegment=chapter, citereset=none}	% part/chapter/section wise references, reset cite label for each segment
		\defbibheading{subbibliography}{\section*{%
				\ifnumgreater{\therefsegment}{0}
				{\nameref{refsegment:\therefsection\therefsegment}}
				{}
			}
		}
	\fi
\fi
%TODO
% possible bugfix for \autocite{} inside of \footnote{text}
%\makeatletter
%\patchcmd\@footnotetext{#1}
%	{\toggletrue{blx@footnote}#1}
%	{\togglefalse{blx@tempa}}{}
%\makeatother

%END_FOLD

% ******************************* Line Spacing ********************************
%BEGIN_FOLD
%---set line spacing with various commands
%\singlespacing
%\onehalfspacing
%\doublespacing

\linespread{1.05}						% factor for  for line spacing

%\setstrech{1.5}						% if setspace package is loaded
%END_FOLD

% ***************************** Table of content ******************************
%BEGIN_FOLD
\setcounter{secnumdepth}{2}
\setcounter{tocdepth}{2}

%---change the name of toc entries
\renewcommand{\contentsname}{Table of contents}
\renewcommand{\listfigurename}{List of figures}
\renewcommand{\listtablename}{List of tables}
\renewcommand{\nomname}{Nomenclature}
\renewcommand{\bibname}{References}
%END_FOLD
%END_FOLD

% *****************************************************************************
% ******************************** Packages ***********************************
% *****************************************************************************
%BEGIN_FOLD
% ********************************* Fonts **************************************
%BEGIN_FOLD
\usepackage[eulergreek]{sansmath}		% to offer sans-serif in mathematics environment in the absence of proper sans maths fonts	
\usepackage{lmodern}
%END_FOLD

% ****************************** Extenstions **********************************
%BEGIN_FOLD
%\usepackage{algpseudocode}
%\usepackage{etex}			% extends Tex towarsd NTS only necesarry for LaTeX distribution older than 2015
\usepackage{import}			% adds \import{full path}{file} used for .pdf_tex files
\usepackage{scrhack}		% solves problems of various packages with KOMA-script; may cause problems when package versions change
\usepackage[utf8]{inputenc}	% set UTF-8 as input 
%\usepackage[utf8]{luainputenc}
\usepackage[T1]{fontenc}	% select font encoding

%END_FOLD

% ******************************* Languages ***********************************
%BEGIN_FOLD
\usepackage[ngerman, english]{babel}	% language support for english and german
\selectlanguage{english}				% select language
\usepackage{csquotes}					% recomended with BibLatTeX for proper quotes
%\usepackage{polyglossia}				% should be used when compiled with XeLatex or LuaTex
%\setmainlanguage{english}
%\setotherlanguage{german}
%END_FOLD

% *************************** Graphics and figures ****************************
%BEGIN_FOLD
%---captions
\setcapindent{0em}									% set caption indent
\setkomafont{captionlabel}{\sffamily\bfseries}		% set caption label sans serif and bold
\addtokomafont{caption}{\small}						% set caption text size
\usepackage{subcaption}								% subcaptions (a), (b), ...

%---float positioning
\usepackage{float}									% fixed float positions with [H] option
\usepackage[section]{placeins}					% force floats to be placed within their respective section
% sets subsections as float barries					
%\makeatletter
%\AtBeginDocument{%
%	\expandafter\renewcommand\expandafter\subsection\expandafter{%
%		\expandafter\@fb@secFB\subsection
%	}%
%}
\makeatother

\usepackage{rotating}								% provies the rotate environment
\usepackage{wrapfig}								% enables figures surounded by text
\usepackage[final]{pdfpages}
%\expandafter\def\csname ver@subfig.sty\endcsname{}	% prevent error when svg and subcaption is loaded
%\usepackage{svg}									% including of .svg files
%TODO svg what for?
%END_FOLD

% ********************************* Colors ************************************
%BEGIN_FOLD
\usepackage{xcolor}
%END_FOLD

% ********************************** TikZ *************************************
%BEGIN_FOLD
\usepackage{tikz}
\include{./Class_Packages_and_Styles/tikzlibraryoptics}
\usepackage[siunitx, nooldvoltagedirection]{circuitikz}	% Loading circuitikz with siunitx option
\usetikzlibrary{positioning, arrows, calc, 3d, decorations.pathreplacing ,decorations.markings, spy, fit, shapes, external} % fadings, shapes

%---custom styles
\tikzset{
	%---term schemes
	level/.style   			= { ultra thick, black},
	virtualLevel/.style		= { ultra thick, dashed, black},
	virtualLevel2/.style	= { ultra thick, dashed, gray},
	connect/.style 			= { ultra thin, black},
	transition/.style		= { thick, black,->,>=stealth',shorten >=1pt},
	transition2/.style		= { very thick, Cornflowerblue,->,>=stealth',shorten >=1pt},
	transitionBoth/.style 	= { thick, black,<->,>=stealth',shorten >=1pt},
	transitionLaser/.style 	= { thick, ->, >=stealth'},
	energyGap/.style		= { thick, black,<->,>=stealth',shorten >=1pt},
	%---dimensioning
	body/.style				= {inner sep=0pt,outer sep=0pt,shape=rectangle,draw,thick,pattern=north east lines wide},
	dimen/.style			= {<->,>=latex,thin,every rectangle node/.style={fill=white,midway,font=\sffamily}},
	symmetry/.style			= {dashed,thin}
}						
\tikzset{>={latex}}					% change default arrows

%---spy setup
\tikzset{
	spy style/.style={#1},
	spy/.style={
		spy scope={
			every spy on node/.style={draw},
			every spy in node/.style={draw},
			spy style
		}
	}
}
\definecolor{spyColor}{HTML}{3949AB}

%---sphereical lens (by Mohammadamin Tajik)
\newcommand\LensSph{} 				% just for safety
\def\LensSph[#1](#2,#3,#4,#5){%
	\pgfmathsetmacro\Tho{asin(0.5*#4/#5)}
	\filldraw[#1]  (#2- .5*#3,.5*#4) arc(-\Tho: \Tho: -#5)-- ++(-#2,0)-- ++(0,#4)--cycle;
}

%Define block styles
\tikzstyle{decision} = [diamond, draw, fill=blue!20, text width=4.5em, text badly centered, node distance=3cm, inner sep=0pt]
\tikzstyle{block} = [rectangle, draw, fill=blue!20, text width=5em, text centered, rounded corners, minimum height=4em]
\tikzstyle{line} = [draw, -latex']
\tikzstyle{cloud} = [draw, ellipse,fill=purple!20, node distance=3cm,minimum height=2em]

%---externalize plots for speed up
\usetikzlibrary{external}							
\tikzexternalize									% externally do the jobs (shell escape needs to be enabled)
\tikzsetexternalprefix{Figures/}					% externalize and save figures in Figures
%---exclude todo pictures from externalize
\usepackage{letltxmacro}
\LetLtxMacro{\oldmissingfigure}{\missingfigure}
\renewcommand{\missingfigure}[2][]{\tikzexternaldisable\oldmissingfigure[{#1}]{#2}\tikzexternalenable}
\LetLtxMacro{\oldtodo}{\todo}
\renewcommand{\todo}[2][]{\tikzexternaldisable\oldtodo[#1]{#2}\tikzexternalenable}
%show plots in draft mode
\pgfkeys{/pgf/images/include external/.code=\includegraphics{#1}}

%---additional pin anchors
%BEGIN_FOLD
%\makeatletter
%\newcommand*\ifStrInTF[2]{%
%	\edef\tikz@temp{{#1}{#2}}%
%	\expandafter\pgfutil@in@\tikz@temp
%	\ifpgfutil@in@\expandafter\pgfutil@firstoftwo\else\expandafter\pgfutil@secondoftwo\fi}
%\newcommand*\ifStrEmptyTF[1]{%
%	\def\tikz@temp{#1}\ifx\tikz@temp\pgfutil@empty
%	\expandafter\pgfutil@firstoftwo\else\expandafter\pgfutil@secondoftwo\fi}
%
%\def\tikz@swapanchor#1.#2\tikz@stop#3#4{#1#4#3}
%\newcommand*\tikzAddAnchor[2]{%
%	\ifStrInTF{.}{#1}{%
%		\ifStrEmptyTF{#2}
%		{\edef#1{\expandafter\tikz@swapanchor#1\tikz@stop{}{}}}
%		{\edef#1{\expandafter\tikz@swapanchor#1\tikz@stop{#2}{.}}}%
%	}{%
%	\ifStrEmptyTF{#2}{}% no true
%	{\edef#1{#1.#2}}%
%}}
%\tikzset{
%	pin anchor/.style={tikz@pin@post/.append style={anchor=#1}},
%	label anchor/.style={tikz@label@post/.append style={anchor=#1}},
%	pin edge pin anchor/.style={
%		append after command={\pgfextra\tikzAddAnchor{\tikzlastnode}{#1}\endpgfextra}%
%	},
%	pin edge parent anchor/.style={
%		append after command={\pgfextra\tikzAddAnchor{\tikz@save@last@node}{#1}\endpgfextra}%
%	}
%}
%\makeatother
%END_FOLD
%END_FOLD

% *********************************** PGF *************************************
%BEGIN_FOLD
\usepackage{pgfplots}
\pgfplotsset{compat=1.8}

%TODO this doesn't fucking work
%---externalize plots for speed up
%\usepgfplotslibrary{external}					% externally do the jobs (shell escape needs to be enabled)
%\tikzexternalize[prefix=./Figures/]				% externalize and save figures in Figures				
%%\tikzset{external/force remake}
%
%%---exclude todo pictures from externalize
%\usepackage{letltxmacro}
%\LetLtxMacro{\oldmissingfigure}{\missingfigure}
%\renewcommand{\missingfigure}[2][]{\tikzexternaldisable\oldmissingfigure[{#1}]{#2}\tikzexternalenable}
%\LetLtxMacro{\oldtodo}{\todo}
%\renewcommand{\todo}[2][]{\tikzexternaldisable\oldtodo[#1]{#2}\tikzexternalenable}
%
%%---global plots settings
%\pgfplotsset{legend style={overlay}, 			% set legends etc. as overlay to horizontaly align accoording to the axes
%			 ylabel style={overlay},
%			 yticklabel style={overlay}}
%\newcommand{\marksSize}{3.5pt}					% size for marks in plots
%\pgfplotsset{									% set san serif font for labels, axes, etc.
%	tick label style = {font=\sansmath\sffamily},
%	every axis label = {font=\sansmath\sffamily},
%	legend style = {font=\sansmath\sffamily},
%	label style = {font=\sansmath\sffamily}
%}

%---zoom area variables for plot zooms with secound axis environment
\newcommand{\xStartZoom}{0.0}
\newcommand{\xEndZoom}{0.0}
\newcommand{\yStartZoom}{0.0}
\newcommand{\yEndZoom}{0.0}
\newcommand{\xStartZoomTwo}{0.0}
\newcommand{\xEndZoomTwo}{0.0}
\newcommand{\yStartZoomTwo}{0.0}
\newcommand{\yEndZoomTwo}{0.0}

%---pgf styles
\pgfplotsset{
	CoilCurrent/.style={line width=1.0pt, thick, y filter/.expression={y==0 ? nan : y},
	filter discard warning=false}
}
\pgfplotsset{
	x axis color/.style={
		xticklabel style=#1,
		xlabel style=#1,
		x axis line style=#1,
		xtick style=#1
	}
}
\pgfplotsset{
	y axis color/.style={
		yticklabel style=#1,
		ylabel style=#1,
		y axis line style=#1,
		ytick style=#1
	}
}

%---add custom plot marks
\pgfdeclareplotmark{xo}{%
	\pgfpathcircle{\pgfpointorigin}{\pgfplotmarksize}%
	\pgfpathmoveto{\pgfqpoint{-1.2\pgfplotmarksize}{-1.2\pgfplotmarksize}}%
	\pgfpathlineto{\pgfqpoint{1.2\pgfplotmarksize}{1.2\pgfplotmarksize}}%
	\pgfpathmoveto{\pgfqpoint{-1.2\pgfplotmarksize}{1.2\pgfplotmarksize}}%
	\pgfpathlineto{\pgfqpoint{1.2\pgfplotmarksize}{-1.2\pgfplotmarksize}}%
	\pgfusepathqstroke%
}
\pgfdeclareplotmark{Y}{%
	\pgfpathmoveto{\pgfpointorigin}%
	\pgfpathlineto{\pgfqpoint{0.0\pgfplotmarksize}{-1.0\pgfplotmarksize}}%
	\pgfpathmoveto{\pgfpointorigin}%
	\pgfpathlineto{\pgfqpoint{-1.0\pgfplotmarksize}{1.0\pgfplotmarksize}}%
	\pgfpathmoveto{\pgfpointorigin}%
	\pgfpathlineto{\pgfqpoint{1.0\pgfplotmarksize}{1.0\pgfplotmarksize}}%
	\pgfusepathqstroke
}

%---discontinuity
\pgfdeclaredecoration{discontinuity}{start}{
	\state{start}[width=0.5\pgfdecoratedinputsegmentremainingdistance-0.5\pgfdecorationsegmentlength,next state=first wave]{}
	\state{first wave}[width=\pgfdecorationsegmentlength, next state=second wave]{
		\pgfpathlineto{\pgfpointorigin}
		\pgfpathmoveto{\pgfqpoint{0pt}{\pgfdecorationsegmentamplitude}}
		\pgfpathcurveto
		{\pgfpoint{-0.25*\pgfmetadecorationsegmentlength}{0.75\pgfdecorationsegmentamplitude}}
		{\pgfpoint{-0.25*\pgfmetadecorationsegmentlength}{0.25\pgfdecorationsegmentamplitude}}
		{\pgfpoint{0pt}{0pt}}
		\pgfpathcurveto
		{\pgfpoint{0.25*\pgfmetadecorationsegmentlength}{-0.25\pgfdecorationsegmentamplitude}}
		{\pgfpoint{0.25*\pgfmetadecorationsegmentlength}{-0.75\pgfdecorationsegmentamplitude}}
		{\pgfpoint{0pt}{-\pgfdecorationsegmentamplitude}}
	}
	\state{second wave}[width=0pt, next state=do nothing]{
		\pgfpathmoveto{\pgfqpoint{0pt}{\pgfdecorationsegmentamplitude}}
		\pgfpathcurveto
		{\pgfpoint{-0.25*\pgfmetadecorationsegmentlength}{0.75\pgfdecorationsegmentamplitude}}
		{\pgfpoint{-0.25*\pgfmetadecorationsegmentlength}{0.25\pgfdecorationsegmentamplitude}}
		{\pgfpoint{0pt}{0pt}}
		\pgfpathcurveto
		{\pgfpoint{0.25*\pgfmetadecorationsegmentlength}{-0.25\pgfdecorationsegmentamplitude}}
		{\pgfpoint{0.25*\pgfmetadecorationsegmentlength}{-0.75\pgfdecorationsegmentamplitude}}
		{\pgfpoint{0pt}{-\pgfdecorationsegmentamplitude}}
		\pgfpathmoveto{\pgfpointorigin}
	}
	\state{do nothing}[width=\pgfdecorationsegmentlength,next state=do nothing]{
		\pgfpathlineto{\pgfpointdecoratedinputsegmentlast}
	}
	\state{final}{
		\pgfpathlineto{\pgfpointdecoratedpathlast}
	}
}
%END_FOLD

% ************************* Maths and Physics Symbols **************************
%BEGIN_FOLD
\usepackage{textcomp}					% registered and copyright symbols \textregistered \textcopyright

%---mathematic packages
\usepackage{amsmath}					% mathematical facileties
\usepackage{amsfonts}					% mathematical fonts
\usepackage{amssymb}					% mathematical symbols
\usepackage{bm}							% bold symbols in maths environment	(not ideal with times)
\usepackage{braket}						% bracket notation

%---units
\usepackage{siunitx} 					% use this package module for SI units
\sisetup{detect-all}					% always used currently set font
\DeclareSIUnit{\gauss}{G}				% Gauss
\DeclareSIUnit{\year}{yr}				% year
\DeclareSIUnit{\eV}{eV}					% electron Volt
\DeclareSIUnit{\atoms}{atoms}			% atom count
\DeclareSIUnit{\mm}{\milli \meter}
%\sisetup{inter-unit-product =$\cdot$}	% for dots as inner product (danger!!! produces Texstudio crash)
%END_FOLD

% ********************************** Tables ************************************
%BEGIN_FOLD
\usepackage{booktabs} 		% For professional looking tables,
\usepackage{multirow}		% to merge cells in Tabular,
%\usepackage{multicol}
%\usepackage{longtable}
%\usepackage{tabularx}
%END_FOLD

% ******************************* Miscallaneous ********************************
%BEGIN_FOLD
\usepackage{comment}		% For \begin{comment} ... \end{comment}
\usepackage{menukeys}		% to display keys strokes or context menu enteries
%END_FOLD

% ******************************** Referencing *********************************
%BEGIN_FOLD
\usepackage{cleveref}				% cleveref for fully dynamic referencing
%\KOMAoptions{footnotes=multiple}

%END_FOLD

% *********************************** Code *************************************
%BEGIN_FOLD
\usepackage{listings}
\definecolor{dkgreen}{rgb}{0,0.6,0}
\definecolor{NumberColor}{HTML}{EF9A9A}
\definecolor{mauve}{rgb}{0.58,0,0.82}
\definecolor{CodeBackgroundColor}{HTML}{FAFAFA}

\lstset{frame=tb,
	language=C++,
	aboveskip=3mm,
	belowskip=3mm,
	showstringspaces=false,
	columns=flexible,
	basicstyle={\small\ttfamily},
	numbers=none,
	numberstyle=\tiny\color{NumberColor},
	keywordstyle=\color{blue},
	commentstyle=\color{dkgreen},
	stringstyle=\color{mauve},
	breaklines=true,
	breakatwhitespace=true,
	tabsize=3,
	stepnumber=2,
	backgroundcolor=\color{CodeBackgroundColor}
}
%END_FOLD

%END_FOLD

% ******************************************************************************
% ************************* User Defined Commands ******************************
% ******************************************************************************
%BEGIN_FOLD
% ********************************* styling ************************************
%BEGIN_FOLD
%---fix for float bug of unwanted indents after floats
\makeatletter
\renewcommand\float@endH{
	\@endfloatbox\vskip\intextsep
	\if@flstyle
		\setbox\@currbox\float@makebox\columnwidth
	\fi
	\box\@currbox\vskip\intextsep\relax\@doendpe
}
\makeatother

%---emphasize names
\newcommand{\person}[1]{{\fontfamily{cmr}\selectfont {\textsc{#1}}}}	% fontfamily for Small Caps

%---for boxes equations
\newlength\dlf															% frame for formulas
\newcommand{\alignedbox}[2]{ 												
	% #1 = before alignment
	% #2 = after alignment
	&\begingroup
	\settowidth\dlf{$\displaystyle #1$}
	\addtolength\dlf{\fboxsep+\fboxrule}
	\hspace{-\dlf}
	\fcolorbox{red}{yellow}{$\displaystyle #1 #2$}
	\endgroup
}

%---fancy closed sqrt
\usepackage{letltxmacro}
\makeatletter
\let\oldr@@t\r@@t
\def\r@@t#1#2{%
	\setbox0=\hbox{$\oldr@@t#1{#2\,}$}\dimen0=\ht0
	\advance\dimen0-0.2\ht0
	\setbox2=\hbox{\vrule height\ht0 depth -\dimen0}%
	{\box0\lower0.4pt\box2}}
\LetLtxMacro{\oldsqrt}{\sqrt}
\renewcommand*{\sqrt}[2][\ ]{\oldsqrt[#1]{#2}}
\makeatother

%---space for - to aling positive and negative numbers
\newcommand{\minusspace}{\hspace{0.82em}}
%END_FOLD

% ********************************* symbols ************************************
%BEGIN_FOLD
\newcommand{\ud}{\: \mathrm{d}}									% upright d for integrals and derivatives	\ud
\newcommand{\ue}{\mathrm{e}}									% upright Exp e								\ue
\newcommand{\expo}[1]{\mathrm{e}^{#1}}							% upright e for exponential function 		\expo{#1}
\newcommand{\ui}{\mathrm{i}}									% upright imaginary i						\ui
\newcommand{\de}[2]{\frac{\mathrm{d} #1}{\mathrm{d} #2}}		% for derivatives		  		 			\de{#1}{#2}
\newcommand{\dd}[2]{\frac{\mathrm{d}^2 #1}{\mathrm{d} #2^2}}	% for double derivatives  	 	 			\dd{#1}{#2}
\newcommand{\pd}[2]{\frac{\partial #1}{\partial #2}}			% for partial derivatives 		 			\pd{#1}{#2}
\newcommand{\pdd}[2]{\frac{\partial^2 #1}{\partial #2^2}}		% for double partial derivatives			\pdd{#1}{#2}
\newcommand{\gv}[1]{\ensuremath{\mbox{\boldmath$ #1 $}}}		% for Greek vectors							\gv{#1}
\newcommand{\uv}[1]{\ensuremath{\bm{\hat{#1}}}}					% for unit vector							\uv{#1}
\newcommand{\abs}[1]{\left| #1 \right|}							% for absolute value						\abs{#1}
\newcommand{\avg}[1]{\left< #1 \right>}							% for average								\avg{#1}
\newcommand{\mean}[1]{\overline{#1}}							% for mean value							\mean{#1}
\newcommand{\entspricht}{\ensuremath{\hat{=}}} 					% for =^									\entspricht
\newcommand{\kf}{\dfrac{1}{4 \pi \epsilon_0}}					% for 1/4pi e0								\kf
\newcommand{\au}{\; \text{(a.u.)}}								% for atomic units							\au
\newcommand{\sollgelten}{\overset{!}{=}}						% for =!									\sollgelten
\newcommand{\electron}{\text{e}^-}								% for e-									\electron
\newcommand{\bohrmag}{\mu_\mathrm{B}}							% for bohr magneton mu_B					\bohrmag
\newcommand{\boltzmann}{k_\mathrm{B}}							% for Boltzmann constant					\boltzmann
\newcommand{\grad}{\vec{\nabla}}								% for vec-nabla sym.						\grad
\newcommand{\Alpha}{\mathcal{L}}								% Lagrangian L								\Alpha
\newcommand{\Beta}{\mathcal{B}}									% "beautiful" B								\Beta
\newcommand{\Order}{\mathcal{O}}								% "beautiful" O								\Order
\newcommand{\Op}[1]{\bm{\expandafter\widehat{#1}}}				% vectorial operator						\Op{#1}
\newcommand{\op}[1]{\expandafter\widehat{#1}}					% operator									\op{#1}
\newcommand{\expect}[1]{\left\langle #1 \right\rangle}			% expected value							\expect{#1}
\newcommand{\nab}{\gv{\nabla}} 									% for nablasymbol							\nab
\newcommand{\divg}[1]{{\mathop{\gv{\nabla}}\nolimits}\cdot #1}	% for divergence							\div{#1}
\newcommand{\rot}[1]{{\mathop{\gv{\nabla}}\nolimits}\times #1}	% for curl									\rot{#1}
\newcommand{\laplace}[1]{{\mathop{\gv{\Delta}}\nolimits} #1}	% for Laplace								\laplace{#1}
\newcommand{\ham}{\op{H}}										% for Hamiltonian							\ham
%END_FOLD

% ***************************** Abbreviations **********************************
%BEGIN_FOLD
\newcommand{\Rb}{$^{87}$Rb }						% Rubidium 87
\newcommand{\RB}{$^{85}$Rb }						% Rubidium 85
%END_FOLD
%END_FOLD

% ******************************************************************************
% ******************************* Definitions **********************************
% ******************************************************************************
%BEGIN_FOLD

%END_FOLD