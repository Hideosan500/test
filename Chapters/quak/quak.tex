% !TeX encoding = UTF-8
% !TeX spellcheck = en_US
% !TeX root = ../../Thesis.tex

\chapter{Short introduction to QUAK}
\label{ch:Short introduction to QuaK}

The goal of quantum klystron (QUAK)\cite{rätzel2020quantum} is to drive Rabi oscillations between the ground state hyperfine levels of the potassium atom using a classical modulated electron beam. Suitable isotopes for this experiment are $^{39}\mathrm{K}$ which has a transition frequency of \SI{461.7}{\mega\hertz} or $^{41}\mathrm{K}$ with an even lower transition frequency of \SI{254}{\mega\hertz} \cite{tiecke:potassium-properties}. For this, it is necessary to \person{Doppler}-cool the atoms in a magneto-optical trap (MOT). Once the atoms are cold and state selected, they will be exposed to the near-field of the electron beam, which will be spatially modulated with the transition frequency. In order to achieve this goal, the beam must allow for a current of \SI{100}{\micro\ampere} and a beam waist of the order of \SI{100}{\micro\meter}.

During our project, we conducted first experiments on how to accomplish an electron beam fulfilling those requirements.