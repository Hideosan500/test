% !TeX encoding = UTF-8
% !TeX spellcheck = en_US
% !TeX root = ../../Thesis.tex

\chapter{Short introduction to QUAK}
\label{ch:Short introduction to QuaK}

The goal of QUAK (quantum klystron) is to drive Rabi oscillations between hyperfine levels of $^{39}\mathrm{K}$ which has a frequency of \SI{461.7}{\mega\hertz} \cite{tiecke:K-frequency} using a classical electron beam. For this, it is necessary to Doppler cool the atoms in a MOT (magneto-optical trap) using lasers. Once the atoms are trapped, they will be exposed to the near-field of the electron beam, which will be spatially modulated with the transition frequency. In order to achieve this goal, the beam must allow for a current of \SI{100}{\micro\ampere}
and a beam waist of \SI{100}{\micro\meter}.

In our work, we conducted first experiments on how to accomplish an electron beam fulfilling those requirements.