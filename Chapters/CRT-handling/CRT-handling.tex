% !TeX encoding = UTF-8
% !TeX spellcheck = en_US
% !TeX root = ../../Thesis.tex

\chapter{CRT handling}
\label{chap:CRT handling}

\section{Opening CRTs}
\label{sec:Opening CRTs}

In order to use hit the $^{39}\mathrm{K}$ cloud with an electron beam, it is necessary to cut open the CRT. This section explains the different methods which were tried and which resulted in clean and easy cuts. All slices were made in a glove box filled with nitrogen gas (\cref{fig:glovebox}) to avoid oxygen poisoning of the cathode.

\begin{figure}[h]
	\missingfigure[figwidth=0.9\textwidth]{Image of glove box.}
	
	\caption{Glovebox filled with nitrogen gas to open CRTs.}
	\label{fig:glovebox}
\end{figure}


\subsection{Rotary tool}
\label{subsec:Rotary tool}


First, a small hole was drilled in the center of the CRT pins to pressurize the CRT with nitrogen. Then a diamond wheel attached to a rotary tool \todo[caption={Dremel trademark}]{\href{https://en.wikipedia.org/wiki/List_of_generic_and_genericized_trademarks}{link}} was used to cut the glass. This method was tried twice, but did not work well the second time. (\todo{Ask Thomas} Why did it not work well? Did the diamond wheel break?) Another obstacle is the plastic box, since it is not fully transparent and therefore made more difficult to see inside. Furthermore, glass dust adhered on the plastic and made it even harder to see the CRT from outside.


\subsection{Wire cutting}
\label{subsec:Wire cutting}

Higher success was achieved by cutting the glass with a heated wire. Two wires were put through the glove box with each inside ending being a ring terminal. A small height adjustable gadget was built out of optical table parts (\cref{fig:Gadget to cut CRT with wire}) in which the CRT was put vertically and looped by a thin wire \todo{wire dimensions, material}. When looping the wire it is important to keep a small gap to avoid an electrical short. Therefore two notches were made in which the wire was fixed.

The assembly was put inside the glove box which was subsequently filled with nitrogen. A current of approximately \SIrange{2}{2.5}{\ampere} was used to heat the thin wire which will result in a breaking point inside the CRT glass. This method does not require a CRT pressurization before the cut. In order to not destroy a device by mistake, this procedure can be first tested on drinking glasses.

\begin{figure}[h]
	\missingfigure[figwidth=0.9\textwidth]{Image of gadget.}
	
	\caption{Gadget to cut CRT with wire.}
	\label{fig:Gadget to cut CRT with wire}
\end{figure}

------------------------------------------------------------------------------

Oxygen poisoning 
Versuchsreihe: Wie schnell degradet die Kathode



2019-12-11

tested helium content in drinking glass by having the bottom open for 30 s, 1 min, 3 min, 6 min, 10 min
lighter goes off when putting it in glass

tested the same with nitrogen with plastic wrap put on opening with rubber band and opening at the top for 3 min, 6 min, 10 min which also works with a lighter (flame goes off)

tested nitrogen with CRT Dev 3 for 6 min, 10 min which also works

tested CRT Dev 3 with helium leak tester
1st test with 1 foil, 1 rubber band
background: 8e-5 mbarl/s
CRT foil/opening next to probe: 2e-4 mbarl/s  to 4e-4 mbarl/s
next to open He gas cylinder: 2e-3 mbarl/s and up to 1.3e-2 mbarl/s

2nd test with 1 foil, 1 rubber band
background: 7e-5 mbarl/s
CRT foil/opening next to probe: 2.8e-4 mbarl/s to 4e-4 mbarl/s
CRT rubber/foil band next to probe: 4e-4 mbarl/s to 8e-4 mbarl/s

3rd test with 3 foils, 2 rubber bands
background: 2.2e-4 mbarl/s
CRT foil/opening next to probe: 2.9e-4 mbarl/s
CRT rubber band below foil: 7e-4 mbarl/s to 1.3e-3 mbarl/s

4th test with 1 aluminum foil hot glued to CRT
background: 6.6e-5 mbarl/s
on glued spot: 7e-5 mbarl/s to 7e-4 mbarl/s mostly under 1e-4 mbarl/s 
on foil itself: 7 mbarl/s to 8.5 mbarl/s