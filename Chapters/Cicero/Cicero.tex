% !TeX encoding = UTF-8
% !TeX spellcheck = en_US
% !TeX root = ../../Thesis.tex

\chapter{Cicero Word Generator}\label{chap:Cicero}
This chapter describes the installation and initial setup of Cicero Word Generator\autocite{keshet2013distributed} on a PC running Windows 10 with analog and digital cards from National Instruments (NI). The code is freely available on Github\autocite{akeshet:Github}. This chapter contains only differences, problems, and possible solutions encountered when Cicero was installed for the PC `Fritz Fantom' which will be used for the QuaK experiment. It is therefore advised to use the technical and user manual\autocite{akeshet:manual} in conjunction. The titles in this chapter and font style with {\fontfamily{pcr}\selectfont Courier} and \textbf{Boldface} was mirrored to fit the manual.

\section{Installation of National Instruments drivers}\label{sec:NI Drivers}
Before setting up the Cicero Word Generator, it is necessary to install the newest .NET Framework\autocite{microsoft:download.net} from Microsoft. For the first installation of NI drivers, NI-DAQmx (version 9.3), NI-VISA (newest version), and NI-4888.2 (newest version) should be downloaded from the National Instruments website\autocite{ni:drivers}. When installing the NI drivers it is possible to get an `Runtime Error!'. In this case it is necessary to set the Regional format settings of Windows 10 to `English (United States)'\autocite{ni:runtimeerror}.

\section{Installation of National Instruments Cards}\label{sec:NI Cards}
After installation of the necessary drivers, the physical cards can be inserted into the PCIe slots on the motherboard. On `Fritz Fantom' the digital card (NI PCIe-6537B) was installed in PCIe bus 3 while the analog cards (NI PCIe-6738) were installed in PCIe bus 4 and 5.

\section{Configuring Atticus}\label{sec:Configuring Atticus}
After installation of the NI cards, Atticus should be launched for the first time and closed without changing any settings. After this, the NI-DAQmx drivers should be updated to the newest version. If version 9.3 was not used when launching Atticus in this step, it could result in an error. After this, ``Configuring Atticus'' on the user manual can be followed. The \textbf{Server Name} was set to `Fritz\_Phantom'\todo[caption={namechange?}]{change server name in lab? Fantom, not Phantom}. \textbf{Dev1} to \textbf{Dev3} were set in the same ascending order as the physical installation on the motherboard.

\subsection{Configure  hardware  timing / synchronization}\label{subsec:hardware  timing/synchronization}
For synchronization, a \textbf{Shared Sample Clock} was used with \textbf{Dev1} being the master card. The settings are summarized in \cref{tab:settings dev1} and \cref{tab:settings dev2}. For \textbf{Dev3} `SampleClockExternalSource' should be set to `/Dev3/RTSI7'. The `SampleClockRate' is set to \SI{350}{\kilo\hertz} since this is the fastest rate with all 32 analog channels active. It is possible to raise this to \SI{1}{\mega\hertz} by only using 8 channels (1 channel per bank).
\begin{table}[H]
	\centering
	\caption{Settings for \textbf{Dev1}.}
	\label{tab:settings dev1}
	\begin{tabular}{*{2}{|l}|}
		\hline
		\textbf{Setting} & Value \\ \hline
		MasterTimebaseSource & \\ \hline
		MySampleClockSource & DerivedFromMaster \\ \hline
		SampleClockRate & 350000 \\ \hline
		UsingVariabletimebase & False \\ \hline
		SoftTriggerLast & True \\ \hline
        StartTriggerType & SoftwareTrigger \\ \hline
	\end{tabular}        	
\end{table}

\begin{table}[H]
	\centering
	\caption{Settings for \textbf{Dev2}.}
	\label{tab:settings dev2}
	\begin{tabular}{*{2}{|l}|}
		\hline
		\textbf{Setting} & Value \\ \hline
		MasterTimebaseSource & \\ \hline
		MySampleClockSource & External \\ \hline
		SampleClockExternalSource & /Dev2/RTSI7 \\ \hline
		SampleClockRate & 350000 \\ \hline
		UsingVariabletimebase & False \\ \hline
		SoftTriggerLast & False \\ \hline
		StartTriggerType & SoftwareTrigger \\ \hline
	\end{tabular}        	
\end{table}

\section{Configuration and Basic Usage of Cicero}\label{sec:Configuration and Basic Usage of Cicero}
After setting up the Atticus server, Cicero can be configured. In step 3.c. it is necessary to write the full IP address and not `localhost'.