% !TeX encoding = UTF-8
% !TeX spellcheck = en_US
% !TeX root = ../../Thesis.tex

\chapter{Cicero Word Generator}\label{chap:Cicero, Cicero Word Generator}
This chapter describes the installation and initial setup of the software Cicero Word Generator \autocite{keshet2013distributed} which is generally used for atomic physics experiments. Cicero uses a client-server architecture with the client Cicero and server Atticus. (Both the full software suite and the user interface client are generally referred to by the same name, Cicero, but the intended meaning is usually clear from context.) Digital and analog sequences can be created for National Instruments (NI) cards. The code is freely available on GitHub \autocite{akeshet:Github}. The program was set up on a PC running Windows 10. This chapter contains only differences, problems, and possible solutions encountered when Cicero was installed for the PC `Fritz Fantom' which will be used for the QUAK experiment. It is therefore advised to use the technical and user manual \autocite{akeshet:manual} in conjunction.

\section{Installation of National Instruments drivers}\label{sec:Cicero, Installation of National Instruments drivers}
Before setting up the Cicero Word Generator, it is necessary to install the newest .NET Framework \autocite{microsoft:download.net} from Microsoft. Then NI drivers, NI-DAQmx (version 9.3), NI-VISA (newest version), and NI-4888.2 (newest version) should be downloaded from the National Instruments website \autocite{ni:drivers} and installed. When installing the NI drivers it is possible to get a `Runtime Error!'. In this case it is necessary to set the Regional format settings of Windows 10 to `English (United States)' \autocite{ni:runtimeerror}.

\section{Installation of National Instruments Cards}\label{sec:Cicero, Installation of National Instruments Cards}
After installation of the necessary drivers, the physical cards can be inserted in the PCIe slots on the motherboard. On `Fritz Fantom' a digital card (NI PCIe-6537B) was installed on PCIe bus 3 while two analog cards (both NI PCIe-6738) were installed on PCIe bus 4 and 5.

\section{Configuring Atticus}\label{sec:Cicero, Configuring Atticus}
After installation of the NI cards, Atticus should be launched for the first time and closed without changing any settings. After this, the NI-DAQmx drivers should be updated to the newest version. If version 9.3 was not used when launching Atticus in this step, it could result in an error. After this, ``Configuring Atticus'' on the user manual can be followed. The \textbf{Server Name} was set to `Fritz\_Fantom'. \textbf{Dev1} to \textbf{Dev3} were set in the same ascending order as the physical installation on the motherboard.

\subsection{Configure hardware timing / synchronization}\label{subsec:Cicero, Configure hardware timing / synchronization}
For synchronization, a \textbf{Shared Sample Clock} was used with \textbf{Dev1} being the master card. The settings are summarized in \cref{tab:settings dev1} and \cref{tab:settings dev2}. \textbf{Dev3} follows the same settings as \textbf{Dev2} except `SampleClockExternalSource' was set to `/Dev3/RTSI7'. The `SampleClockRate' is set to \SI{350}{\kilo\hertz} since this is the fastest rate with all 32 analog channels active. It is possible to raise this to \SI{1}{\mega\hertz} by only using 8 channels (1 channel per bank).

To route the clock signal to the timing bus, under `Connections' the `Destination Terminal' was set to \textbf{/Dev1/RTSI7} and `Source Terminal' to \textbf{/Dev1/DO/SampleClock}.
\begin{table}[H]
	\centering
	\caption{Settings for \textbf{Dev1}.}
	\label{tab:settings dev1}
	\begin{tabular}{*{2}{|l}|}
		\hline
		\textbf{Setting}      & Value             \\ \hline
		MasterTimebaseSource  &                   \\ \hline
		MySampleClockSource   & DerivedFromMaster \\ \hline
		SampleClockRate       & 350000            \\ \hline
		UsingVariabletimebase & False             \\ \hline
		SoftTriggerLast       & True              \\ \hline
		StartTriggerType      & SoftwareTrigger   \\ \hline
	\end{tabular}        	
\end{table}

\begin{table}[H]
	\centering
	\caption{Settings for \textbf{Dev2}.}
	\label{tab:settings dev2}
	\begin{tabular}{*{2}{|l}|}
		\hline
		\textbf{Setting}          & Value           \\ \hline
		MasterTimebaseSource      &                 \\ \hline
		MySampleClockSource       & External        \\ \hline
		SampleClockExternalSource & /Dev2/RTSI7     \\ \hline
		SampleClockRate           & 350000          \\ \hline
		UsingVariabletimebase     & False           \\ \hline
		SoftTriggerLast           & False           \\ \hline
		StartTriggerType          & SoftwareTrigger \\ \hline
	\end{tabular}        	
\end{table}

\section{Configuration and Basic Usage of Cicero}\label{sec:Cicero, Configuration and Basic Usage of Cicero}
After setting up the Atticus server, Cicero can be configured. In step 3.c. it is necessary to write the full IP address and not `localhost'.

\subsection*{Saving of Settings and Sequences}\label{sec:Cicero, Saving of Settings and Sequences}
The `SettingsData' of the Server Atticus are saved in C:\textbackslash Users\textbackslash confetti\textbackslash Documents \textbackslash Cicero\textunderscore Atticus\textbackslash Cicero\textbackslash SettingsData while the `SequenceData' of Cicero are saved in C:\textbackslash Users\textbackslash confetti\textbackslash Documents\textbackslash Cicero\textunderscore Atticus\textbackslash Cicero\textbackslash SequenceData.

\subsection*{Sequence length limit}\label{sec:cicero, Sequence length limit}
The duration of a sequence is limited to $\num[exponent-base=2]{e32}/(16*32*\SI{350}{\kilo\hertz})$ = \SI{23.967}{\second} coming from a 32-bit application, \SI{16}{\bit} per channel, 32 channels in a NI PCIe-6738 card, and \SI{350}{\kilo\hertz} clock rate.