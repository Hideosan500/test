% !TeX encoding = UTF-8
% !TeX spellcheck = en_US
% !TeX root = ../../Thesis.tex

\ctikzset{sources/scale=0.5}
\ctikzset{transformer L1/.style={inductors/coils=6}, transformer L2/.style={inductors/coils=6}}  % set coil number so that midtap touches coil

\newcommand{\coaxial}[2] % coordinates of left inner wire, length of cable
% draw a coaxial cable with given length
{
	\draw (0 + {#1}[0], {#1}[1]) to[short, -] (#2 + {#1}[0], {#1}[1]); % middle line
	\draw (0.5 + {#1}[0], 0.25 + {#1}[1]) -- (#2 - 0.5 + {#1}[0], 0.25 + {#1}[1]); % top line
	\draw (0 + {#1}[0], -0.25 + {#1}[1]) to[short, -] (#2 + {#1}[0], -0.25 + {#1}[1]); % bottom line
	\draw (0.5 + {#1}[0], {#1}[1]) circle [radius=0.25]; % left circle
	\draw (#2 - 0.5 + {#1}[0], {#1}[1]) circle [radius=0.25]; % right circle
}

%\draw [help lines] (0, 0) grid (15, 12);
%\filldraw (0, 0) circle (0.5mm);


% signal generator
\draw (0, 3) rectangle (2, 4) node [midway]{RF};
\coaxial{2, 3.5}{2}
\draw (4, 3.25) node [ground] {};
\draw (4, 3.5) to ++ (1, 0);

% x signal
\draw (5, 3.5) to ++ (0, 1.5) to ++ (2, 0) to [twoport, t=AMP] ++ (2, 0) node [transformer, anchor=A1] (x-transformer) {};
\draw (x-transformer.A2) node [ground] {};  % gound left side
\draw (x-transformer.B1) to ++ (2, 0) node [right] {x1}; % connection to x1
\draw (x-transformer.B2) to ++ (2, 0) node [right] {x2}; % connection to x2
\draw (x-transformer-L2) to [dcvsource, label=\SI{96}{\volt}] ++ (2, 0) node [ground] {};  % bias center to ground

% y signal
\draw (5, 3.5) to ++ (0, -2) to [vphaseshifter] ++ (2, 0) to [twoport, t=AMP]++ (2, 0) node [transformer, anchor=A1] (y-transformer) {};
\draw (y-transformer.A2) node [ground] {};  % gound left side
\draw (y-transformer.B1) to ++ (2, 0) node [right] {y1}; % connection to y1
\draw (y-transformer.B2) to ++ (2, 0) node [right] {y2}; % connection to y2
\draw (y-transformer-L2) to [dcvsource, label=\SI{78}{\volt}] ++ (2, 0) node [ground] {};  % bias center to ground
