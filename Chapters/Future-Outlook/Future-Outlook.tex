% !TeX encoding = UTF-8
% !TeX spellcheck = en_US
% !TeX root = ../../Thesis.tex

\chapter{Next Steps}
 
To conclude this report, we will point out some of the next steps, that need to be taken in order to advance the electron beam setup. 

\begin{description}
	\item[Beam current stability:] The most important challenge at this moment is the ability to generate a reproducible, stable, and sufficiently strong beam. In order to achieve this, more research on cathodes and their susceptibility to oxygen poisoning needs to be conducted. It may also prove useful to add another high voltage power supply to the setup, in order to tune the filament potential independently from the Wehnelt cylinder. 
	\item[Spot size characterisation:] As previously mentioned, our original attempt was, to probe the electron beam's profile using a thin piece of wire on the wobble stick. However, we have observed that the beam got warped, when passing close to conductive materials. Therefore, a different approach is needed. 
	\item[Heating mechanism:] As high currents will degrade the filament and cathode, it is desirable to be able to tune the heater current down to \SI{0}{\milli\ampere} continuously. Such a power supply needs to support a bias voltage of around \SI{-2}{\kilo\volt}. The current heater (described in section \cref{sec:Heater}) has a minimal current of approximately \SI{60}{\milli\ampere}. It could be improved by using a potentiometer with a larger resistance range or a variable transformer. 
	\item[Lissajous Curves:] Regarding the deflection electronics, the first issue that needs to be addressed, is the fact that it is not possible to produce a clean sine wave, when a bias voltage is applied to the center tapped transformer. Furthermore, it is recommended to implement the over voltage protection described previously with very low capacitance diodes.  
	In the future, the setup should be able to produce Lissajous curves at the  $^{39}\mathrm{K}$ hyperfine transition frequency of \SI{461.7}{\mega\hertz}. 

\end{description}